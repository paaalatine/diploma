\mysection*{ЗАКЛЮЧЕНИЕ}
\

В рамках данной работы были выполнены следующие задачи:
\begin{itemize}
  \item Выполнен анализ области машинного обучения, задачи классификации, машины опорных векторов, что позволило использовать эту информацию для реализации программы; 
  \item Определены функциональные и нефункциональные требования к программе, на основе которых было составлено техническое задание и были выбраны инструменты и средства разработки;
  \item Разработана архитектура программы в соответствии с поставленными требованиями;
  \item Реализована программа в соответствии с разработанной архитектурой, которая показала точность классификации более 95\% и время ответа не более 0.01с. Программа также не устанавливает связи, если в запросе содержались несуществущие названия моделей и номера. Модули обучения и классификации работают независимо друг от друга, что позволяет изпользовать обученную модель многократно. Присутствует возможность обращения к классификатору через API, что делает возможным интеграцию в какое-либо приложение;
  \item Программа была протестирована, что позволило убедиться в её соответстивии установленным требованиям, а также повысило уверенность в качестве;
\end{itemize}

Результатом данной работы является созданная программа поиска аналогов на основе машинного обучения.

Преполагается дальнейшее развитие проекта и его поддержка, добавление нового функционала и исправление возможных ошибок, которые могут быть найдены во время эксплуатации приложения.

\newpage