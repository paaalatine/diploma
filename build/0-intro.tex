\mysection*{ВВЕДЕНИЕ}

\textbf{Актуальность темы}. В конце 20-го века с развитием информационных технологий стало возможным делать покупки не выходя из дома, возможности для этого предоставляют интернет-магазины. С ростом количества интернет-магазинов появились и системы мониторинга цен и предложений конкурентов для предпринимателей, владеющих интернет-магазинами. Такой мониторинг позволяет автоматически формировать
цены на товары в магазине в соответствии с текущей ситуацией на рынке. Одной из важнейших задач таких систем является установления связей между товарами предпринимателя и его конкурентов. Автоматизация подобных процессов позволяют людям сэкономить время и ресурсы. 

\textbf{Целью работы} является создание программы поиска аналогов на основе машинного обучения. Такая программа сможет без ручного внесения в базу устанавливать связи между товарами владельца интернет-магазина и его конкурентов. Таким образом, нет необходимости отслеживать появление новых товаров у конкурентов и заносить новую связь в базу, программа найдет соответсвие сама, обучившись на большом наборе подобных связей.

Для достижения цели исследования был сформулирован следующий ряд \textbf{задач}:

\begin{itemize}
  \item провести обзор машинного обучения;
  \item определить требования к программе;
  \item осуществить проектирование и разработку программы;
  \item произвести тестирование приложения.
\end{itemize}

\textbf{Объем и структура работы}. Работа содержит 40 страниц печатного текста, 15 рисунков, 2 таблицы, список литературы, включающий 18 источников. Работа состоит из введения, четырех частей и заключения. Во введении обоснована актуаль6ность работы, определены цель и задачи исследования. В первой части произведено краткое описание области машинного обучения, а также обзор существующих решений. Вторая часть содержит описание структуры программы и выбранных средст разработки. Третья глава описывает реализацию программы. Четвертая глава описывает тестирование разработанной программы и харатеристики работы. В заключении приведены основные результаты работы и возможные направления для дальнейшего развития.

\newpage
