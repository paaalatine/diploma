
\mysection{ТЕСТИРОВАНИЕ РАЗРАБОТАННОЙ ПРОГРАММЫ}
\

Тестирование — процесс исследования, испытания программного продукта, является одним из этапов разработки любого программного обеспечения. Оно необходимо для выявления ошибок во время разработки и после её завершения, проверки соответствия между ожидаемым поведением программного обеспечения и реальным, а также оценки качества программного обеспечения. Другими словами, тестирование — это одна из техник контроля качества программного продукта, включающая в себя действия по планированию работ (Test Management), проектированию тестов (Test Design), выполнению тестов (Test Execution) и анализу полученных результатов (Test Analysis).

Качество программного обеспечения (Software Quality) — это способность программного продукта удовлетворять предъявляемым к нему требованиям. 

Основные цели тестирования:

\begin{itemize}
	\item Повышение вероятности правильной работы при любых обстоятельствах.
	\item Повышение вероятности соответствия всем предъявленным к продукту требованиям.
	\item Предоставление актуальной информации о состоянии продукта на данный момент.
\end{itemize}

Для тестирования программы поиска был создан отдельный модуль.

\newpage

\subsection{Тестирование программы}
\

При тестирования программы использовалось модульное тестирование. Модульное тестирование — процесс, который позволяет проверить на корректность отдельные модули. Такое тестирование позволяет точно определить местонахождение ошибки, так как компоненты тестируются независимо друг от друга. Модульные тесты позволяют проверить, не привело ли очередное изменение кода к регрессии (появление ошибок в частях программы, которые уже были протестированы).

С помощью данного вида тестирования проверялись все методы разработанных классов, отражающие функционал сохранения, загрузки, выявления несуществующих названий моделей.

Также были протестированы некоторые неоднозначные случаи, в которых программа должна была повести себя определенным образом. Например, название модели конкурента может иметь пробел в середине, а название модели владельца магазина может быть написано слитно, или наоборот, но программа должна определить, что это одна и та же модель. Также было проверено соответствие функциональному требованию о том, что товары с несуществующими у владельца названиями моделей и идентификационными номерами не должны находиться.

\newpage

\subsection{Тестирование модели}
\

Под тестированием модели в данной работе подразумевается выявление её характеристик работы и проверка соответствия значений этих характеристик установленным ранее требованиям.

Скользящий контроль или кросс-проверка или кросс-валидация (cross-validation, CV) — процедура эмпирического оценивания обобщающей способности алгоритмов, обучаемых по прецедентам. 

Исходный набор данных разделяется на две подвыборки: обучающую и контрольную. Для каждого разбиения выполняется построение модели на основе обучающей подвыборки, затем оценивается её средняя ошибка на объектах контрольной подвыборки. Оценкой кросс-валидации называется средняя по всем разбиениям величина ошибки на контрольных подвыборках.

Скользящий контроль является стандартной методикой тестирования и сравнения алгоритмов классификации, регрессии и прогнозирования. 

С помощью кросс-валидации определялась точность предсказаний, из всего корпуса данных выделялось 20\% в качестве контрольной подвыборки, 80\% в качестве обучающей. Модель была обучена несколько раз на разном количестве документов в обучающей коллекции. При этом каждый раз замерялось и сохранялось время обучения. Далее для каждого случая модель классифицировала 100 разных документов и среднее время ответа было определено как время работы модели.

Результаты описанных выше действий представлены в таблице 4.1. На основании этих данных можно сделать следующие выводы:

\begin{itemize}
	\item Характеристики времени работы и точности классификации удовлетворяют заданным требованиям;
	\item Время обучения нелинейно растет при увеличении количества документов в корпусе данных;
	\item Характеристики точности и времени работы слабо варьируются для разного количества документов в рамках данного теста.
\end{itemize}

\begin{table}[h!]
\caption{Характеристики программы}
\label{props}
\centering
    \begin{tabular}{|c|c|c|c|}
		\hline Корпус, пар & Время обучения, с & Время работы, с & Точность, \% \\
		\hline 50000 & 384,1475 & 0.0003 & 96 \\
		\hline 20000 & 79.8404 & 0.0001 & 98 \\
		\hline 10000 & 11.7035 & 0.0001 & 98 \\
		\hline 5000 & 2.5664 & 0.0001 & 97 \\
		\hline 1000 & 0.3084 & 0.0001 & 98 \\
		\hline
	\end{tabular}
\end{table}
\

\newpage

