\mysection{РАЗРАБОТКА АРХИТЕКТУРЫ ПРОГРАММЫ}
\
\subsection{Требования к реализуемой программе}
\

Требования к программам и системам можно разделить на функциональные и нефункциональные. Функциональные требования позволяют обозначить функционал, которым должна обладать программа для удовлетворения просьб заказчика. Нефункциональные требования – это дополнительные атрибуты качества, которые важны для разработчика и заказчика.

К разрабатываемой программы были поставлены следующие функциональные требования:

\begin{itemize}
  \item Программа должна иметь функционал для сохранения и загрузки обученной модели;
  \item Программа не должна находить связь, если в запросе был указан товар, который не существует в базе у клиента;
  \item Предсказания должны осуществляться с точностью более 90\%;
  \item Время ответа классификатора не должно превышать 0.1 с.
\end{itemize}

\
Нефункциональные требования:
\begin{itemize}
  \item Программа должна иметь API для возможности интеграции в приложение.
\end{itemize}

\newpage

\subsection{Проектирование архитектуры}
\


\subsubsection{Проектирование модуля обучения}
\
\subsubsection{Проектирование модуля предсказаний}
\
\subsubsection{ПРоектирование API}

\newpage