\newgeometry{
  top=20mm,
  right=15mm,
  bottom=20mm,
  left=20mm,
  bindingoffset=0cm
}

\thispagestyle{empty}

\begin{center}
  {
    \bfseries
    {
      \subnormal
      Министерство науки и высшего образования Российской Федерации
    } \\[-0.5em]
    {
      \scriptsize
      ФЕДЕРАЛЬНОЕ ГОСУДАРСТВЕННОЕ АВТОНОМНОЕ ОБРАЗОВАТЕЛЬНОЕ УЧРЕЖДЕНИЕ ВЫСШЕГО ОБРАЗОВАНИЯ
    } \\[-0.25em]
    {
      \subnormal
      “САНКТ-ПЕТЕРБУРГСКИЙ НАЦИОНАЛЬНЫЙ ИССЛЕДОВАТЕЛЬСКИЙ \\[-0.5em]
      УНИВЕРСИТЕТ ИНФОРМАЦИОННЫХ ТЕХНОЛОГИЙ, \\[-0.75em]
      МЕХАНИКИ И ОПТИКИ” \\[2em]
    }
  }
\end{center}

\small

\begin{flushright}
  \begin{minipage}{.5\textwidth}
    {
      \hfill\textbf{УТВЕРЖДАЮ}\hfill
    }

    \titledline{Руководитель ОП}

    \setlength{\remaining}{\textwidth}\addsignatureskip
    $\underset{\text{\scriptsize (Фамилия, И.О.)}}{\underline{\makebox[\remaining][s]{\strut\hfill}}}$
    \hfill\signature \\[-0.5em]

    \hfill\datetemplate \\[0.35em]
  \end{minipage}
\end{flushright}

\begin{center}
  {
    \bfseries
    {
      \normalsize
      З А Д А Н И Е \\
    }
    НА  ВЫПУСКНУЮ  КВАЛИФИКАЦИОННУЮ  РАБОТУ \\[1.5em]
  }
\end{center}

{
  \parindent0pt

  \textbf{Студенту}
  $\underline{\text{\strut Кануковой С. А.~~}}$
  \hfill
  \textbf{Группа}
  $\underline{\text{\strut P3402~~}}$
  \hfill
  \textbf{Факультет}
  $\underline{\text{\strut ПИиКТ~~}}$ \\[-0.5em]

  \titledline{\textbf{Руководитель ВКР}}
  $\underset{
    \text{\scriptsize (ФИО, ученое звание, степень, место работы, должность)}
  }{
    \underline{\makebox[\remaining][s]{\strut Тропченко Андрей Александрович, к.т.н., доцент, Университет ИТМО\hfill}}
  }$ \\[-0.5em]

  \textbf{1 Наименование темы}
  \uline{Разработка программы поиска аналогов на основе машинного обучения\hfill} \\[-1em]
  
  \titledline{\textbf{Направление подготовки (специальность)}}
  $\underline{
    \makebox[\remaining][s]{\strut 09.03.01 -- Информатика и вычислительная техника\hfill}
  }$ \\[-1em]

  \titledline{\textbf{Направленность (профиль)}}
  $\underline{
    \makebox[\remaining][s]{\strut Вычислительные машины, комплексы, системы и сети\hfill}
  }$ \\[-1em]

  \titledline{\textbf{Квалификация}}
  $\underline{
    \makebox[\remaining][s]{\strut бакалавр\hfill}
  }$ \\[-1em]

  \textbf{2 Срок сдачи студентом законченной работы}\hfill\datetemplate \\[-1em]

  \textbf{3 Техническое задание и исходные данные к работе} \\
  \uline{
    Разработать программу поиска аналогов, провести тестирование и апробацию.\hfill
  }\\
  \uline{
    Исходные данные:\hfill
  }\\
  \uline{
    - набор данных для обучения модели\hfill
  }\\
  \uline{
  Требования к программе:\hfill
  }\\
  \uline{
  - результат поиска с точностью выше 90\%\hfill
  }\\
  \uline{
  - программа не должна находить аналоги, которые не содержат ключевые слова (идентификацион-\hfill
  }\\
   \uline{
   ный номер, название модели), указанные при поиске\hfill
  }\\[-1em]
}

\restoregeometry

\clearpage

\newgeometry{
  top=20mm,
  right=20mm,
  bottom=20mm,
  left=15mm,
  bindingoffset=0cm
}

\thispagestyle{empty}

{
  \parindent0pt

  \textbf{4 Содержание выпускной квалификационной работы (перечень подлежащих разработке} \\
  \textbf{вопросов)}\\
  \uline{
    4.1 Обзор предметной области;\hfill
  }\\
  \uline{
    4.2 Выбор инструментов разработки;\hfill
  }\\
  \uline{
    4.3 Описание процесса разработки;\hfill
  }\\
  \uline{
    4.4 Апробация результатов разработки.\hfill
  }\\[-1em]

  \textbf{5 Перечень графического материала (с указанием обязательного материала)} \\
  \uline{
    Презентация по проделанной работе (в формате PDF)\hfill
  }\\
  \uline{
    Слайд №1 "Постановка задачи"\hfill
  }\\
  \uline{
    Слайд №2 "Архитектура программы"\hfill
  }\\
  \uline{
    Слайд №3 "Характеристики программы"\hfill
  }\\
  \uline{
    Слайд №4 "Примеры работы программы"\hfill
  }\\[-1em]

  \textbf{6 Исходные материалы и пособия} \\
  \uline{
    6.1 Флах, Петер Машинное обучение. Наука и искусство построения алгоритмов, которые извлека-\hfill
  }\\
  \uline{
    ют знания из данных. Учебник // Петер Флах. - М.: ДМК Пресс. 2015.\hfill
  }\\
  \uline{
    6.2 Батура Т.В. Методы автоматической классификации текстов // Программные продукты и системы. 2017. №1. - Режим доступа: https://cyberleninka.ru/article/n/metody-avtomaticheskoy-klassifikatsii-\hfill
  }\\
  \uline{
    tekstov\hfill
  }\\
  \uline{
    6.3 Scikit-learn Documentation [Электронный ресурс] // Официальный сайт проекта Scikit-learn. 2007-2019. - Режим доступа: https://scikit-learn.org/stable/documentation.html\hfill
  }\\[-1em]

  \textbf{7 Дата выдачи задания} \datetemplate\\[-1em]

  Руководитель ВКР~~\signature\\[-1em]

  Задание принял к исполнению~~\signature\hfill\datetemplate\\
}

\normalsize
\restoregeometry

\clearpage
