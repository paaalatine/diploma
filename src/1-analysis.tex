\mysection{АНАЛИЗ ПРЕДМЕТНОЙ ОБЛАСТИ}
\
\subsection{Машинное обучение. Задача классификации}
\

\textbf{Машинное обучение} – группа методов искусственного интеллекта, которые используются компьютерными системами для эффективного выполнения конкретной задачи без использования явных инструкций, вместо этого обучаясь на множестве решений сходных задач.

Машинное обучение подразделяется на обучение по прецедентам, что означает выявление общих закономерностей по частным эмпирическим данным, и дедуктивное обучение, при котором формируется база знаний на основе формализованных знаний экспертов.

В данной работе речь пойдет об обучении по прецедентам, одним из подтипов которого является обучение с учителем. Обучение с учителем – самый распространенный случай, оно производится на наборе обучающих примеров, каждый из которых представлет собой пару "объект, ответ". В процессе обучения выводится функциональная зависимость ответа от объекта и строится алгоритм, который позволяет отображать входные данные в выходные. Когда множество возможных ответов конечно, говорят о задачах классификации и распознавания образов.

\textbf{Задача классификации} – ответ принадлежит конечному множеству, он называется меткой класса. Класс представляет собой множество объектов, которые соответсвуют данному значению метки. Задано конечное множество объектов, для которых известно, к каким классам они относятся. Это множество называется выборкой. Классовая принадлежность остальных объектов неизвестна. Требуется построить алгоритм, способный классифицировать произвольный объект из исходного множества.

\textbf{Классификатором} называется отображение $\widehat{c}: \mathcal{X} \to \mathcal{C}$, где 
\
$\mathcal{C} = \{C_{1}, C_{2}, \dots, C_{k}\}$ — конечное множество меток классов. Под $C_{i}$ можно также понимать множество объектов, которые относятся к классу с номером $i$. Знак "крышки" означает, что $\widehat{c}(x)$ — оценка истинной, но неизвестной функции $c(x)$. Обучающие примеры для классификатора имеют вид пар $(x,c(x))$, где $x \in \mathcal{X}$ — объект, а $c(x)$ — истинный класс, к которому принадлежит этот объект. Под обучением классификатора понимается выявление функции $\widehat{c}$, которая как можно лучше аппроксимирует $c$ не только на обучающем наборе, но в идеале и на всем пространстве объектов.\cite{MLFLACH}

\

\textbf{Типы классификации:}
\begin{itemize}
  \item Бинарная классификация – требуется определить к какому из двух классов относится объект, простой в техническом отношении случай, который служит основой для решения более сложных задач;
  \item Многоклассовая классификация – более двух классов, задача классификации становится более трудной, решающая граница не столь очевина, как в первом случае;
  \item Непересекающиеся классы – объект относится только к одному из множества классов;
  \item Пересекающиеся классы – объект может относиться одновременно к нескольким классам;
  \item Нечёткие классы – определяется степень принадлежности объекта каждому из классов.
\end{itemize}

\textbf{Типы входных данных:}
\begin{itemize}
  \item Признаковое описание — каждый объект – это набор признаков, признаки могут быть числовыми или нечисловыми;
  \item Матрица расстояний между объектами. Каждый объект описывается расстояниями до всех остальных объектов обучающей выборки. С этим типом входных данных работают методы ближайших соседей, парзеновского окна, потенциальных функций;
  \item Временной ряд или сигнал представляет собой последовательность измерений во времени;
  \item Изображение или видеоряд.
\end{itemize}

Также входные данные могут представляться в виде графов, текстов. Они приводятся к первому или второму типу путём предварительной обработки данных и извлечения признаков.

\

\subsection{Анализ имеющихся решений}
\

Одним из имеющихся решений является Elasticsearch. \textbf{Elasticsearch} - свободная программная поисковая система, написанная на языке Java, которая используется большим числом крупных сайтов и компаний, например, GitHub, Foursquare, SoundCloud и другие. Данная система может использоваться для решений многих задач, одной из которых является сопоставление товаров для систем мониторинга цен конкурентов.\cite{ELASTIC} Однако для организации связей товаров Elasticsearch требует четкого задания классов синонимов, то есть групп слов, которые будут распознаваться как один и тот же признак в названии товара. Такой способ не подходит, если в поиске участвует большой объем названий товаров из разных категорий и невозможно вручную установить связи между словами из-за их количества и неопределенности.

\newpage
\subsection{Постановка задач исследования}
В рамках данной работы поставлена задача разработать программу, которая сможет находить связи между товарами, основываясь на предсказаниях обученного на корпусе данных классификатора.
Для решения поставленной цели необходимо решить следующие задачи:
\begin{itemize}
  \item Определить требования к программе;
  \item Выбрать модель для обучения;
  \item Спроектировать архитектуру программы;
  \item Реализовать программу в соответствии с разработанной архитектурой;
  \item Провеси тестирование.
\end{itemize}
\newpage
